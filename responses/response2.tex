\documentclass[12pt]{article}
\usepackage[margin=1in]{geometry}
\usepackage{setspace}
\usepackage{amsmath}

\usepackage[margin=1in]{geometry}
\usepackage[american]{babel}
\usepackage{csquotes}
\usepackage{hyperref}
\hypersetup{colorlinks=true,urlcolor=blue,citecolor=blue}
\usepackage[hyperref=true,style=apa,backend=biber,isbn=false,natbib=true]{biblatex}
\DeclareLanguageMapping{american}{american-apa}
\AtEveryCitekey{\clearfield{month}}

\bibliography{../_bibliography/references}

\newcommand{\question}[1]{
    \noindent\textbf{#1}

    \doublespacing
}

\newcommand{\references}[0]{\newpage\singlespacing\printbibliography}

\begin{document}

% \question{What is ``simulation''? Discuss the various ways that the concept of simulation has been used in explanations or models of the mind, and argue for the approach that you think is most promising. Why is this approach the most promising? On this approach, what kind of evidence would reveal that a particular cognitive process does or does not involve simulation?}

\question{What is the relationship between simulation and probabilistic models of cognition? Specifically, one interpretation is that simulations might be thought of as samples from a probability distribution. Does this interpretation work for all the ways in which simulation has been used to explain cognition? Discuss why or why not.}

\subsection*{Introduction}

In cognitive science, the notion of \textit{simulation} has been invoked to explain a vast variety of psychological phenomena. In probabilistic models of cognition, simulation (specifically, probabilistic simulation) has also been suggested as a process-level mechanism for approximating Bayesian inference. Are the notions of simulation in cognitive science related to the notion of simulation as approximate Bayesian inference? In general, I will argue that the answer is ``yes''; however, there are certain cases where I argue that the \textit{simulator} is a sample from a probability distribution, rather than the simulations themselves.

Here, I will summarize the way that simulation has been appealed to in cognition, and for each version of simulation, discuss how related that notion of simulation is to the idea of simulation as Bayesian inference. Broadly, I have identified three general areas in which simulation is said to occur: low-level sensorimotor prediction, mid-level conceptual grounding, and high-level reasoning with ``runnable'' mental models.

\subsection*{Simulation as sensorimotor prediction}

When the term ``simulation'' is used in the context of cognitive science, most people will likely think of the \textit{simulation theory} \citep{Gallese1998}. The simulation theory was originally a hypothesis about social cognition \citep{Gordon1992,Goldman1992}, and gained popularity in the late 90s and 2000s with the discovery of mirror neurons \citep{Gallese1998}. Mirror neurons, originally discovered in monkeys, have the interesting behavior that they fire both when the monkey performs an action as well as when the monkey observes a \textit{different} monkey performing the same action. This behavior has been taken to be a sign that the mind \textit{simulates} what its own behavior would be in order to understand and interpret the behavior of others. This idea has been quite popular and has been suggested to play a role in many other domains (for example, speech perception, see \cite{Fischer2008}).

In motor control, there is strong evidence that the motor system develops \textit{forward models} of its own dynamics \citep{Kawato1999,Flanagan2003}. For example, \cite{Flanagan2003} ran an experiment in which people had to move a block from one point to another; however, they modified the dynamics of the block such that it had an upwards velocity proportional to the horizontal velocity. People had to learn to adjust both the grip on the block exerted by their fingers, and the overall force exerted by their arm and hand. \cite{Flanagan2003} showed that people learn to adjust the forces exerted by their fingers first, suggesting that they first updated a model of forward sensory prediction governing their finger grip before updating an inverse model of control for the arm and hand.

Another line of research into the phenomenon of \textit{displacement} (also known as \textit{representational momentum}) suggest some process of forward predictive models \citep{Freyd1984,Freyd1988,Hubbard2005}. \cite{Freyd1984} showed that, when asked to determine whether two successive images were the same, people had more difficulty discriminating two different images when the second image was displaced in the direction of an implied motion (e.g., forward motion) than when it was displaced in the other direction. The hypothesis for these results was that people had a distorted memory in the direction of motion, as if they had made a prediction about the next location of the object and stored that in memory rather than the original percept. This finding has since been replicated many times, and has been shown to occur not just for moving objects but also for static objects that would be accelerated by external forces \citep{Freyd1988}. Many experiments have also demonstrated displacement effects involving angular velocity, friction, rotation, barriers, shape, knowledge of physical properties such as mass, etc. \citep{Hubbard2005}. These results together suggest that the perceptual system makes sophisticated predictions about the next state of an object based on relevant physical factors that may be involved.

One hypothesis that unifies these various modes of sensorimotor prediction is the \textit{emulation theory} of perception \citep{Grush2004}. In this theory, \cite{Grush2004} proposes that the mind builds models both of the behavior of its own body as well as of its percepts, and uses these models in control, perception, and imagery. Importantly, he distinguishes his hypothesis of \textit{emulation} from the simulation theory. In the emulation theory, simulations are generated by running the controller with the learned forward dynamics model. In the simulation theory, the controller is merely detached from the true dynamics; motor commands are issued by the controller but they do not do anything because they are not actually ``hooked up'' to a limb during the simulation.

How do the types of low-level simulation suggested by simulation theory and emulation theory compare to probabilistic simulation? I would argue that, under the interpretation from emulation theory, low-level sensorimotor prediction is closely related to probabilistic simulation in the context of \textit{filtering}. \cite{Grush2004} explicitly makes the connection to Kalman filters which are a statistical method for performing inference over Hidden Markov Models. Particle filters are another type of filtering algorithm which rely on Monte-Carlo sampling to estimate the posterior and have already been used (albeit in other contexts) as a rational process model \citep{Abbott2011}. Under the interpretation from simulation theory, I think the connection to probabilistic models of cognition--and to sampling in particular--is less clear. Like Grush, it is not clear to me how issuing fake motor commands aids in prediction or inference without positing something else that the commands are issued \textit{to}.\footnote{I think the interpretation is clearer in the simulation theory of social cognition, but I will come back to this later in the section on ``runnable'' mental models.}

\subsection*{Simulation as conceptual grounding}

At a slightly higher level of cognition we find \textit{mental simulation} used as an explanation for conceptual understanding, particularly in language comprehension. This use of mental simulation is usually automatic and people may be unaware of its presence most of the time (though, perhaps, not \textit{entirely} unaware).

\cite{Matlock2004} suggest that mental simulation is engaged when words conveying \textit{fictive motion} are used. Fictive motion is not real motion, but instead is evoked when a motion-related word (usually a verb) is used to describe something inanimate. For example, \textit{The road runs through the desert} involves fictive motion because roads cannot physically ``run''. In the study, participants read stories involving long or short distance travel, fast or slow motion, and rough or easy terrain. They then read a sentence involving fictive motion. \cite{Matlock2004} found that participants took longer to respond whether a sentence involving fictive motion was related to the story or not when the motion was related to a long/slow/rough story than for the short/fast/easy stories. Her suggestion was that, perhaps, participants constructed some mental model of the story and then simulated from that mental model in order to determine whether the sentence was related.

In a similar vein, \cite{Bergen2007} looked at whether mental simulation was involved in language comprehension by investigating the \textit{linguistic Perky effect}. The original Perky effect demonstrated a link between visual mental imagery and visual perception: participants stared at a blank screen while mentally visualizing an object (e.g., a hammer) while an image of that object was gradually displayed on the screen (beginning with a projection which was imperceptible); several participants apparently did not realize that they were gazing at a real image and thought it was their mental visualization. This effect has subsequently been used to show that mental imagery can interfere with visual perception. Thus, \cite{Bergen2007} asked a similar question of imagery in language comprehension: if a sentence implies movement in a particular direction, are people slower and less accurate at identifying a visual cue in that direction? They found positive evidence for that hypothesis; for example, people were slower and more inaccurate at identifying cues at the top of the screen (as opposed to the bottom of the screen) when they heard a sentence like \textit{The balloon rose}.

Can the type of mental simulation used in language comprehension be thought of as samples from a probability distribution? Under the \textit{perceptual symbol systems} theory, \cite{Barsalou1999} would likely say no, because probability distributions are not grounded. I would argue, however, that the \textit{representations} used by probability distributions could certainly be the sort of perceptual symbols advocated for by Barsalou. The combination of these representations into a coherent interpretation of the utterance can then be thought of as a particular instantiation of a structured hypothesis, sampled from a probability distribution. Thus, the simulation itself is a sample from a distribution over interpretations.

\subsection*{Simulation as a ``runnable'' mental model}

Besides the mirror neuron simulation theory, the idea of a ``runnable'' mental model \cite{Craik1943} is perhaps the next most pervasive view of simulation in cognitive science. I will argue that the runnable mental model view of simulation encompasses many different aspects of higher-level cognition, including mental imagery, mental models, model theory, theory of mind, and thought experiments.

In 1971, \cite{Shepard1971} showed that people engaged in a form of \textit{mental simulation} when determining whether two images depicted the same shape (which differed by a rotation) or different shapes (which differed by a rotation and a reflection). Since then, mental imagery has been studied extensively, most notably by \cite{Kosslyn2006}. It has been shown that mental imagery can be used as a tool for creative visual imagination \citep{Finke1988}; that it can be used for reasoning about mechanical models \citep{Hegarty2004}; that there is motor imagery as well as visual imagery \citep{Parsons1994,Flusberg2011}; and that motor imagery may engage a type of ``dynamic'' imagery involving physical constraints, as opposed to ``kinematic'' imagery which is purely perceptual \citep{Schwartz1999a}. Given that there is a clear deliberative component involved in mental imagery--i.e, one can choose what to visualize and where to visualize it--one might interpret mental imagery as being a type of simulation from a mental model.

\cite{Gendler1998} argues that mental models--specifically ones that involve tacit knowledge about how the world works--are key to scientific thought experiments. This notion of mental models is closely tied to the notion of mental imagery discussed in the previous paragraph, especially that discussed by \cite{Schwartz1999a}. \cite{Gendler1998} argues that it is exactly \textit{because} thought experiments tap into otherwise inaccessible tacit knowledge about the world that they are useful constructs in science. Because the knowledge is otherwise inaccessible, such mental model-based thought experiments provide something new that one would otherwise not be able to deduce. \cite{Clement2009} gives similar arguments and analyzes the use of mental model-based thought experiments by experts while reasoning about the behavior of a physical system; he finds that experts spontaneously use such thought experiments and that they are crucial to the reasoning process. Similarly, \cite{Trickett2007} show that that scientists frequently rely on informal thought experiments in the process of scientific reasoning in their own domain and that such thought experiments are used to try to explain data that inconsistent with their hypotheses.

The types of mental models used in thought experiments may rely on physical intuitions such as those investigated by \cite{Schwartz1999a}, but may also be more conceptual in nature. Indeed, there is evidence that without a conceptual model of the domain, people (e.g., novices, or experts outside of their domain of expertise) will rely on their naive physical intuitions--perhaps engaging in dynamic imagery--while domain experts will construct a mental model from more abstract concepts \citep{Gentner1983}. These types of mental models tend to be more qualitative in nature, and have been investigated from a modeling perspective through the use of qualitative simulation \citep{Kuipers1986,Forbus2011}.

Mental models of a different flavor have also been proposed by \cite{Johnson-Laird2012} to explain logical reasoning. In the \textit{model theory}, as it is known in this context, mental models encode propositions in a particular way (for example, that is biased towards expressing things that are true, rather than things that are false). People then ``simulate'' from these mental models as they engage in the process of logical reasoning; this notion of simulation, however, is not entirely clear to me. Perhaps more well-defined is the notion of simulation in \textit{kinematic} mental models \citep{Khemlani2013}, which posits a way in which deduction and abduction is used in algorithmic thinking. Here, it is clearer that running a simulation involves the construction and execution of an algorithm based on the constraints of the problem. In this sense, kinematic mental models are similar to the mental models discussed above.

Finally, as mentioned briefly earlier, simulation has been proposed as a theory of social cognition \citep{Gordon1992,Goldman1992}. Under this theory, people take their decision-making centers ``offline'' and run them with pretend inputs in order to determine what another person would do, or why a person is acting in the way that they are acting. By default, the inputs are the same as the ones that one would use in their own decision making process \citep{Gordon1992}, but may be modified as necessary to reflect differences in situation between oneself and the other person. The simulation theory has been criticized as not providing very much explanatory power, and moreover, for being inconsistent with both developmental and neural data \citep{Stich1992,Gopnik1992,Saxe2005}.

In the preceding paragraphs, I have discussed the phenomenon of mental imagery and have argued that it is closely tied to the idea of the ``runnable'' mental model. Indeed, I have suggested that many types of simulation in higher-level cognition fall under the umbrella of the ``runnable mental model'' (with the exception, perhaps, of the simulation theory of mind). How are mental imagery and runnable mental models related to probabilistic models of cognition? I would suggest that static mental imagery (or the initial frame of dynamic mental imagery) can be interpreted as samples from a probability distribution. In particular, the creative visual synthesis demonstrated by \cite{Finke1988} seems especially similar to moving around in probability space according to a sampling mechanism like MCMC (similar to the Captcha inference algorithm developed by Mansinghka et al.). I would also suggest that runnable mental models themselves can be thought of as a particular type of structured hypothesis analogous to a program (such as those constructed by \cite{Khemlani2013}). Constructing a mental model, then, is related to sampling from the distribution over hypotheses; however, I do not see the \textit{running} of the mental model as being a sampling procedure in and of itself (though it could be, depending on whether the mental model is deterministic or stochastic).

What of the simulation theory of mind? I am inclined to say that the co-opting of an existing decision making process does not constitute probabilistic simulation unless the decision making process itself uses probabilistic simulation (which it certainly may). However, I am more inclined to agree with the theory theorists interpretation of theory of mind \citep{Gopnik1992}. Even assuming that the simulation theory is correct in terms of the decision process being co-opted, I think that the it pushes the hardest questions out of the way: how does one choose what ``pretend'' inputs to use? When does someone differ enough from oneself to change from the ``default''? Arguably, to make these types of decisions about how to construct the pretend inputs, one would require a theory of beliefs, desires, and goals that specifies how important various differences are to specify. In this sense, a theory can be thought of as a probabilistic model \citep{Griffiths2009,Baker2014}. If that theory produces such pretend inputs, which are then used to simulate expected behavior, then yes: this type of simulation can be thought of as a sample from a probability distribution. An even more preferable interpretation over co-opting the decision process is something in between the simulation theory and the theory theory \citep{Saxe2005}. This alternative might, perhaps, look more like the emulation model theory of mind \citep{Grush2004}: people construct a forward model of behavior which they use to produce predictions and an overlying theory of mind is used to actually determine the controls the forward model.

\references
\end{document}
