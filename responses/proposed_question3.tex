\documentclass[12pt]{article}
\usepackage[margin=1in]{geometry}
\usepackage{setspace}
\usepackage{amsmath}

\usepackage[margin=1in]{geometry}
\usepackage[american]{babel}
\usepackage{csquotes}
\usepackage{hyperref}
\hypersetup{colorlinks=true,urlcolor=blue,citecolor=blue}
\usepackage[hyperref=true,style=apa,backend=biber,isbn=false,natbib=true]{biblatex}
\DeclareLanguageMapping{american}{american-apa}
\AtEveryCitekey{\clearfield{month}}

\bibliography{../_bibliography/references}

\newcommand{\question}[1]{
    \noindent\textbf{#1}

    \doublespacing
}

\newcommand{\references}[0]{\newpage\singlespacing\printbibliography}

\begin{document}

\question{What is ``simulation''? Discuss the various ways that the concept of simulation has been used in explanations or models of the mind. How are these uses of simulation similar or different? Is there any core insight that can be gleaned from the intersection of these topics?}

\subsection*{Introduction}

The term ``simulation'' has been used by many researchers over many decades as a description for how cognition works. But, what exactly \textit{is} simulation? Given that it has been used everywhere from the context of low-level neural activity \citep{Gallese1998} to high-level scientific thought experiments \citep{Gendler1998}, it is not at all clear what the term ``simulation'' means on its own. I have identified three broad areas in which simulation has been appealed to as an explanation for cognition: low-level sensorimotor prediction, mid-level conceptual grounding, and high-level ``runnable'' mental models. I will argue that, despite their surface dissimilarities, these three areas are closely related through their relationship to perception and action.

\subsection*{Simulation as sensorimotor prediction}

These low-level simulations are governed by our low-level sensorimotor systems and are not cognitively penetrable.

\begin{itemize}
\item Neural simulation
    \begin{itemize}
    \item Mirror neurons \citep{Gallese1998}
    \item ``Communicative'' motor resonance \citep{Fischer2008}
    \end{itemize}
\item Sensorimotor prediction
    \begin{itemize}
    \item Representational momentum / displacement \citep{Freyd1984,Freyd1988,Hubbard2005}
    \item Emulation theory of perception \citep{Grush2004}
    \item Forward and inverse models in motor control \citep{Kawato1999,Flanagan2003}
    \item Model of gravity \citep{Zago2005}
    \item Perception of force \citep{White2012a}
    \item Action-oriented predictive processing \citep{Clark2013}
    \end{itemize}
\item Generative models and analysis-by-synthesis
    \begin{itemize}
    \item Helmholtz machine \citep{Helmholtz1925,Dayan1995}
    \item Speech recognition \citep{Halle1962,Halle1959,Bever2010}
    \item Vision \citep{Yuille2006}
    \end{itemize}
\end{itemize}

\subsection*{Simulation as conceptual grounding}

These mid-level simulations occur in day-to-day experiences (e.g. spontaneously visualizing events in a story you are reading) and we may not always be aware of these types of simulations. If we pay attention and introspect, we might be more aware of them.

\begin{itemize}
\item Language comprehension
    \begin{itemize}
    \item Linguistic Perky effect \citep{Bergen2007}
    \item Fictive motion \citep{Matlock2004}
    \item ``Referential'' motor resonance \citep{Fischer2008}
    \end{itemize}
\item Theory of mind
    \begin{itemize}
    \item Simulation theory \citep{Goldman1992,Gordon1992,Gallese1998}
    \item Theory theory \citep{Stich1992,Gopnik1992,Saxe2005}
    \end{itemize}
\item Grounded/embodied simulations \citep{Barsalou1999}
\end{itemize}

\subsection*{Simulation as ``runnable'' mental models}

These high-level simulations are typically constructed in a deliberative manner yet frequently seem to be grounded in perception and action (for example, we cannot change the direction of gravity in our mental simulations \citep{Schwartz1999a}, consistent with reports that the sensorimotor system has a fixed representation of gravity \citep{Zago2005}).

\begin{itemize}
\item Imagery
    \begin{itemize}
    \item Visual imagery \citep{Kosslyn2006,Kosslyn1988,Shepard1971}
    \item Motor imagery \citep{Parsons1994}
    \item Dynamic imagery \citep{Schwartz1999a,Flusberg2011}
    \end{itemize}
\item Imagination
    \begin{itemize}
    \item Creativity \citep{Finke1988}
    \item Image construction \citep{Kosslyn1988}
    \item Imagination of episodic events \citep{Schacter2012}
    \item Simulation heuristic \citep{Kahneman1981}
    \end{itemize}
\item Deliberative reasoning
    \begin{itemize}
    \item Model theory \citep{Johnson-Laird2012,Khemlani2013}
    \item Mental models \citep{Craik1943,Gentner1983,Hegarty2004}
    \item Qualitative simulation \citep{Kuipers1986,Forbus2011}
    \item Scientific thought experiments \citep{Gendler1998,Trickett2007,Clement2009,Brown2014}
    \end{itemize}
\end{itemize}

\references
\end{document}
